\documentclass[10pt]{beamer}

%\usepackage{lmodern}
%\usepackage[labelformat=empty,font=scriptsize,skip=0pt,justification=justified,singlelinecheck=false]{caption}

%\usepackage{paralist}
%\usepackage{amsmath}% http://ctan.org/pkg/amsmath
%\usepackage{amsfonts}% http://ctan.org/pkg/amsfonts

%\usepackage[font=scriptsize]{caption}

%\usepackage{hyperref}
\usepackage{amssymb,amsmath,amsfonts,eurosym,geometry,graphicx,caption,color,setspace,
comment,footmisc,caption,pdflscape,array}
\usepackage{booktabs}   % for nice tables
\usepackage{multirow}

\usepackage{mathtools}

%\usepackage[round]{natbib}
\setbeamertemplate{caption}[numbered]
\usepackage[export]{adjustbox}

\usepackage[skip=1pt]{caption}
%\usepackage[capposition=top]{floatrow}

%\usepackage[caption = false]{subfig}
%\usepackage{floatrow}
\usepackage[capposition=bottom]{floatrow}


%\usepackage{enumitem}%allow alphatebical ordering in enumerate
\usepackage{graphicx}
\usepackage{tabularx}
%\usepackage{threeparttable}
\usepackage{float}
\usepackage{mwe}
%\usepackage{subfig}
%\usepackage{polyglossia}
\usepackage{subcaption}
\setlength{\abovecaptionskip}{2pt}
%\usepackage[tight,TABTOPCAP]{subfigure}
\usepackage[round]{natbib}

\usepackage{multicol, latexsym, amsmath, amssymb}

\usepackage[normalem]{ulem}
\useunder{\uline}{\ul}{}
\usepackage{booktabs,caption}
\usepackage[flushleft]{threeparttable}

\usepackage{graphics}

\usepackage{longtable}

\usepackage{float}

\usepackage{amsbsy} %boldsymbol

%%in case of outdated TEX Live
\usepackage{lmodern}

\usepackage{appendixnumberbeamer}

%\graphicspath{{figures/}{../figures/}{D:/Presentations\figures/}}
\usepackage[normalem]{ulem}

\mode<presentation> {

% The Beamer class comes with a number of default slide themes
% which change the colors and layouts of slides. Below this is a list
% of all the themes, uncomment each in turn to see what they look like.

%\usetheme{default}
%\usetheme{AnnArbor}
%\usetheme{Antibes}
%\usetheme{Bergen}
%\usetheme{Berkeley}
%\usetheme{Berlin}
%\usetheme{Boadilla}
%\usetheme{CambridgeUS}
%\usetheme{Copenhagen}
%\usetheme{Darmstadt}
%\usetheme{Dresden}
%\usetheme{Frankfurt}
%\usetheme{Goettingen}
%\usetheme{Hannover}
%\usetheme{Ilmenau}
%\usetheme{JuanLesPins}
%\usetheme{Luebeck}
\usetheme{Madrid}
%\usetheme{Malmoe}
%\usetheme{Marburg}
%\usetheme{Montpellier}
%\usetheme{PaloAlto}
%\usetheme{Pittsburgh}
%\usetheme{Rochester}
%\usetheme{Singapore}
%\usetheme{Szeged}
%\usetheme{Warsaw}

% As well as themes, the Beamer class has a number of color themes
% for any slide theme. Uncomment each of these in turn to see how it
% changes the colors of your current slide theme.

%\usecolortheme{albatross}
%\usecolortheme{beaver}
%\usecolortheme{beetle}
%\usecolortheme{crane}
%\usecolortheme{dolphin}
%\usecolortheme{dove}
%\usecolortheme{fly}
%\usecolortheme{lily}
%\usecolortheme{orchid}
%\usecolortheme{rose}
%\usecolortheme{seagull}
%\usecolortheme{seahorse}
%\usecolortheme{whale}
%\usecolortheme{wolverine}

%\setbeamertemplate{footline} % To remove the footer line in all slides uncomment this line
%\setbeamertemplate{footline}[page number] % To replace the footer line in all slides with a simple slide count uncomment this line

%\setbeamertemplate{navigation symbols}{} % To remove the navigation symbols from the bottom of all slides uncomment this line
}
\usecolortheme{seahorse}

\usepackage{graphicx} % Allows including images
\usepackage{booktabs} % Allows the use of \toprule, \midrule and \bottomrule in tables

\usepackage{arydshln} %can use hdashline

\setbeamertemplate{footnote}{%
  \hangpara{2em}{1}%
  \makebox[2em][l]{\insertfootnotemark}\footnotesize\insertfootnotetext\par%
}
%----------------------------------------------------------------------------------------
%	TITLE PAGE
%----------------------------------------------------------------------------------------

\title[Research]{Research Concept} % The short title appears at the bottom of every slide, the full title is only on the title page

\author{Eduard Storm} % Your name
\institute[estorm@carleton.edu]
 % Your institution as it will appear on the bottom of every slide, may be shorthand to save space
{
	
	
	\medskip 
	
	
%	Department of Economics \\  
%	Carleton College \\ % Your institution for the title page
%	\textit{estorm@carleton.edu} % Your email address
	
%	\bigskip
	
%	 Job Market Paper Presentation for: \\
%		\smallskip
%	EBS University of Business and Law
}


\date{January 2021} % Date, can be changed to a custom date

\begin{document}

\begin{frame}
\titlepage % Print the title page as the first slide
\end{frame}

%----------------------------------------------------------------------------------------
%	PRESENTATION SLIDES
%----------------------------------------------------------------------------------------

\begin{frame} 
	\frametitle{Near future (1-2 years)}

\textbf{Keywords:} Task data, workplace heterogeneity, gains from specialization, automation, COVID

\bigskip
	
\begin{enumerate}
	\item \textbf{Wrap up \& new corporations}
		\begin{itemize}
			\item Task Specialization in longitudinal setting
			\item Decompose wage inequality with an emphasis on occupation-specific task variability using IAB data (possibly applied to impact of Minimum wage)
			\item Gains from specialization based on \# of tasks performed (Open-Economy; Specialist vs. Generalist)
			\smallskip
			\item Product variety and task composition (Juan)
			\item Patents \& Tasks (w/ Sahar)
		\end{itemize}
\end{enumerate}	


\bigskip

\begin{enumerate}
	\item \textbf{Start COVID-related projects}
	\begin{itemize}
		\item Automation \& inequality (older workers \& youth, Task, IFR, European Companies Survey)
		\item Matching \& remote work (matching gains due to reduced search cost; job vacancy data)
		\item Policy Evaluation: Worker reallocation in response to lockdown policies (Europe: SHARE, GER: TBD - county data)
	\end{itemize}
\end{enumerate}
	
\bigskip

$\Longrightarrow$ Apply for research grant and assemble team	
	
\end{frame}
%------------------------------------------------

\begin{frame} 
	\frametitle{Intermediate future (3+ years)}
	
\textbf{Keywords:} Task data, skill formation, mobility, structural modeling, Big data

\bigskip	
	
\begin{enumerate}
	\item \textbf{Human Capital Formation}
	\begin{itemize}
		\item Cognitive vs non-cognitive (SOEP \& task measures; locus of control)
		\item General purpose vs specific skills (skill match; National Educational Panel Study)
		\item Genoeconomics (conceptualize inherent ability; nature vs nurture in life-cycle settings)
		\item COVID-response \& locus of control
	\end{itemize}
\end{enumerate}

\bigskip

\begin{enumerate}
	\item \textbf{The Political Economy of Globalization \& Technological Change}
	\begin{itemize}
		\item Portability of skills across cohorts (all job vacancy data and spatial analysis makes sense)
		\item Occupational mobility \& political radicalization (also study impact on voter turnout in elections and aftermath)
		\item Optimal migration policies
	\end{itemize}
\end{enumerate}	




	
	
\end{frame}
%------------------------------------------------

\end{document}